% Motivate the problem, situation or topic you decided to
% work on. Describe why it matters (is it of societal, eco-
% nomic, scientific value?). Outline the rest of the paper (use
% references, e.g. to Section 2: What kind of data you are
% working with, how you analyse it, and what kind of conclu-
% sion you reached. The point of the introduction is to make
% the reader want to read the rest of the paper.

% Motivate the problem, situation or topic you decided to work on.
This discrepancy, as highlighted in the analysis by 
\citet{Jasilionis2023} in "\textit{The underwhelming German life expectancy}," 
poses critical questions about the underlying factors contributing to this 
phenomenon.

% Describe why it matters (is it of societal, economic, scientific value?).


% Outline the rest of the paper (use references, e.g.~to \Cref{sec:methods}: 
% What kind of data you are working with, how you analyse it, and what kind of 
% conclusion you reached. The point of the introduction is to make the reader want 
% to read the rest of the paper.

